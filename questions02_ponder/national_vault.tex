\question{}

มีตู้เซฟอยู่ตู้หนึ่ง ตู้เซฟนี้ถูกล็อกด้วยหมายเลขปริศนา \lstinline{Q} ซึ่งมีความยาว 6 หลัก 
(เราเรียก \lstinline{Q} ว่า\emph{\hrsp รหัสผ่านจริง\hrsp})

ทุก ๆ ครั้งที่เราป้อนรหัสเซฟเพื่อเปิดตู้เซฟตู้นี้ หากเราป้อนรหัสไม่ถูกต้อง ตู้เซฟจะมีเสียงร้องพร้อมทั้งยังมีข้อความตอบกลับ
(response message) ว่า
\begin{quote}
    \centering
    เลขโดดที่อยู่ติดกันที่ยาวที่สุดที่ปรากฏทั้งในรหัสผ่านจริง \lstinline{Q} \\
    และ\emph{\hrsp รหัสผ่านที่ป้อนผิด\hrsp}นั้นมีความยาวเท่าใด\hrsp%
    \sidenote{%
        พูดอีกนัยหนึ่งคือ response message จะเป็นความยาวของ 
        \textbf{Longest common substring} (\lstinline{LCS})
        ระหว่างรหัสผ่านจริง \lstinline{Q} กับ\emph{\hrsp รหัสผ่านที่ป้อนผิด\hrsp}
        
        \smallskip
        \textbf{ตัวอย่าง}\; สมมติว่ารหัสผ่านจริงของตู้เซฟคือ 
        \begin{center}
            \lstinline{Q = 123456} 
        \end{center}
        แต่เราป้อนรหัสตู้เซฟเป็น
        \begin{center}
            \lstinline{A = 134579}
        \end{center}
        แล้วตู้เซฟจะมี response message ตอบกลับออกมาเป็น
        \begin{center}
            \lstinline{LCS(A, Q) = 3}
        \end{center}
    }
\end{quote} 

ต่อไปนี้คือประวัติของการลองป้อนรหัสเซฟแก่ตู้เซฟนี้ทั้งสิ้น 9 ครั้ง พร้อมทั้ง response message ในแต่ละครั้ง
\begin{center}
    \smallskip
    \begin{tabular}{@{\quad}c@{\qquad\qquad}c@{\quad}}
        \toprule
        รหัสผ่านที่ป้อน \lstinline|A| & ข้อความตอบกลับ \lstinline|LCS(A, Q)|  \\
        \midrule
        \verb|027292| & \verb|1| \\
        \verb|135135| & \verb|0| \\
        \verb|257015| & \verb|2| \\
        \verb|362447| & \verb|1| \\
        \verb|470619| & \verb|3| \\
        \verb|560968| & \verb|1| \\
        \verb|674669| & \verb|1| \\
        \verb|822642| & \verb|1| \\
        \verb|903287| & \verb|3| \\
        \bottomrule
    \end{tabular}
    \smallskip
\end{center}

จงหารหัสผ่านจริง \lstinline{Q} ของตู้เซฟนี้จากข้อมูลข้างต้น
