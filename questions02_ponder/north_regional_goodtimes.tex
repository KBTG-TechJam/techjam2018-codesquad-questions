\question{}

กำหนดให้มีนาฬิกาดิจิทัลอยู่เรือนหนึ่ง มีลักษณะเป็น 24-hour clock
ที่แสดงผลในรูปแบบ \texttt{HH:MM:SS} ตั้งแต่เวลา \texttt{00:00:00} ไปจนถึง \texttt{23:59:59}

\medskip\noindent
\textbf{\uline{นิยาม}}\;
เวลา ณ วินาทีหนึ่ง ๆ จะเป็น ``เวลาเลขสวย'' ก็ต่อเมื่อการแสดงผลบนหน้าปัดนาฬิกา%
ประกอบไปด้วยเลขโดดที่แตกต่างกัน\uline{ไม่เกิน 2 ตัว}เท่านั้น ยกตัวอย่างเช่น
\begin{itemize}[before*=\small]
\item \texttt{13:31:11} เป็นเวลาเลขสวย เนื่องจากแสดงผลได้ด้วยเลขโดด 1 และ 3 เพียงสองตัว
\item \texttt{11:11:11} เป็นเวลาเลขสวย เพราะใช้เลขโดดเพียงตัวเดียว (ยังไม่เกิน 2 ตัว)
\item \texttt{23:00:00} ไม่เป็นเลขสวย เนื่องจากต้องใช้ตัวเลขโดดถึง 3 ตัวในการแสดงผล
\end{itemize}

\noindent
\textbf{\uline{โจทย์}}\;
จงหาว่า ในช่วงเวลาตั้งแต่ \texttt{01:30:00} ไปจนถึง \texttt{21:00:00} 
นาฬิกาดิจิทัลจะปรากฏเวลาเลขสวยทั้งหมดกี่ครั้ง?
