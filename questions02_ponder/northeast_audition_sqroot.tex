\question{\label{q:ponder_northeast_audition_sqroot}}

จงพิจารณาโปรแกรมดังต่อไปนี้\,\sidenote[][6.15\baselineskip]{%
   \textbf{หมายเหตุ}\; \lstinline{floor} 
   คือฟังก์ชันที่ปัดเศษของจำนวนเต็มทิ้งให้กลายเป็นจำนวนเต็มที่มากที่สุดที่น้อยกว่าหรือเท่ากับจำนวนเดิม
}
\begin{lstlisting}
function mystery_<%\ref*{q:ponder_northeast_audition_sqroot}%>(n):
   # n <%\codecmt เป็นจำนวนเต็มที่ไม่ติดลบ%>
   lb := 0
   ub := n
   loop:
      attempt := floor((lb + ub) / 2)
      if n < attempt^2:
         ub := attempt - 1
      elseif n >= (attempt + 1)^2:
         lb := attempt + 1
      else: 
         break loop and return attempt
      end
   end
end
\end{lstlisting}
โปรแกรมข้างต้นทำหน้าที่ตามที่ระบุในข้อใดต่อไปนี้?

\begin{itemize}[label={$\Circle$}]
\item ฟังก์ชัน square root แต่ปัดเศษเป็นจำนวนเต็มที่ใกล้ที่สุดเสมอ\\ (round to nearest integer)
\item ฟังก์ชัน square root แต่ปัดเศษทิ้งเป็นจำนวนเต็มเสมอ (round down)
\item ฟังก์ชัน square root แต่ปัดเศษขึ้นเป็นจำนวนเต็มเสมอ (round up)
\item ฟังก์ชัน square root แต่เศษอาจถูกปัดขึ้นหรือลงอย่างไรก็ได้ ไม่สามารถคาดเดาได้
\item ฟังก์ชันติด infinite loop ไม่รู้จบ
\end{itemize}
