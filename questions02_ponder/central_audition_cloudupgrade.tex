\question{}

สมมติว่านายกสิกรมีโปรแกรมที่เขียนขึ้นมาเองอยู่โปรแกรมหนึ่ง เพื่อใช้ประมวลผลข้อมูลขนาดใหญ่มหาศาล
ซึ่งโปรแกรมนี้อาจจะต้องใช้เวลาทำงาน\uline{ต่อเนื่อง}หลายชั่วโมง 
และ\uline{ไม่อาจคาดเดาได้}ว่าโปรแกรมนี้จะใช้เวลาประมวลผลกี่ชั่วโมง (จนกว่าโปรแกรมจะรันเสร็จสิ้นเท่านั้น)

นายกสิกรตัดสินใจใช้บริการคลาวด์แห่งหนึ่งเพื่อรันโปรแกรมของตัวเอง 
โดยคลาวด์ดังกล่าวมีนโยบายการคิดค่าบริการดังนี้
\marginnote{
    \textbf{ตัวอย่าง}\; ลองพิจารณาสถานการณ์สมมติดังต่อไปนี้
    \begin{itemize}
    \item 
        สมมติว่านายกสิกรเริ่มนำโปรแกรมนี้ไปรันในคลาวด์นี้แบบคิดค่าเช่ารายชั่วโมง \:\adforn{62}\: 
        เมื่อเวลาผ่านไป 4 ชั่วโมงพบว่าโปรแกรมนี้ยังประมวลผลไม่เสร็จ 
        นายกสิกร จึงตัดสินใจเลือก upgrade เป็น Flat rate \:\adforn{62}\:
        แต่จากนั้นเมื่อเวลาผ่านไปอีก 2 ชั่วโมงโปรแกรมจึงรันเสร็จสิ้น
    
        จึงเท่ากับว่าค่าใช้จ่ายที่ถูกที่สุดที่เป็นไปได้ในทางทฤษฎี (Optimal cost) ในกรณีนี้คือ 7,200 บาท 
        แต่นายกสิกรต้องเสียเงินจริง (Actual cost) ไปถึง 14,800 บาท
        ซึ่งคิดเป็น 2.056 เท่าของ Optimal cost
    
    \item 
        ในอีกเหตุการณ์หนึ่ง สมมติว่านายกสิกรรันโปรแกรมดังกล่าวจนครบ 6 ชั่วโมง
        พบว่ายังประมวลผลไม่เสร็จสิ้น จึงลองเสี่ยง upgrade เป็น Flat rate ดู \:\adforn{62}\:
        แต่สุดท้ายแล้วโปรแกรมนี้ใช้เวลาถึง 20 ชั่วโมงในการรันจนเสร็จสิ้น
    
        จึงเท่ากับว่านายกสิกรเสีย Actual cost ไป 17,200 บาท  
        ซึ่งคิดเป็น 1.72 เท่าของ Optimal cost ที่เกิดจากการเหมาจ่ายตั้งแต่แรกที่ 10,000 บาท
    \end{itemize}
}
\begin{itemize}
\item หากเช่าเป็นชั่วโมง คิดชั่วโมงละ 1,200 บาท
\item ระหว่างที่โปรแกรมของนายกสิกรกำลังรันอยู่และค่าใช้บริการถูกคิดเป็นค่าเช่ารายชั่วโมงอยู่นั้น\;    
    นายกสิกรสามารถเลือก upgrade บริการคลาวด์ให้คิดค่าบริการแบบ Flat rate เมื่อใดก็ได้
    โดยคิดเหมาจ่ายในราคา 10,000 บาท และจะ\uline{ไม่ได้ค่าเช่ารายชั่วโมงก่อนหน้า} 
    \uline{นั้นคืน} (กล่าวคือ upgrade เร็วย่อมคุ้มค่ากว่า upgrade ช้า)
\end{itemize}

\noindent
เมื่อพิจารณาตัวอย่างสถานการณ์ที่ปรากฏทางด้านข้างแล้ว สังเกตว่าปัญหามีอยู่สองส่วนคือ
\begin{enumerate}
\item นายกสิกร ไม่สามารถคาดเดาระยะเวลาที่ Program จะใช้ประมวลผลได้ล่วงหน้า
\item ถ้าเรา upgrade เร็วหรือช้าเกินไป ค่าใช้จ่ายในกรณี worst-case
        อาจจะสูงเกินกว่าที่ควรจะเป็น อันเนื่องมาจากสาเหตุข้อแรก
\end{enumerate}

นายกสิกรต้องการต้องการคิดกลยุทธ (Strategy) เพื่อวางแผนใช้บริการคลาวด์ดังกล่าว
ให้คุ้มค่าทุกบาททุกสตางค์มากที่สุดเท่าที่เป็นไปได้ แม้ว่านายกสิกรจะคาดเดาจำนวนชั่วโมงที่โปรแกรมจะใช้รันไม่ได้เลย\;
กล่าวคือค่าใช้จ่ายจริง (Actual cost) จะต้องมีปริมาณน้อยที่สุดเมื่อ\uline{เทียบอัตราส่วน}%
กับค่าใช้จ่ายที่ถูกที่สุดที่เป็นไปได้ (Optimal cost)

อยากทราบว่านายกสิกรควรวางแผนเช่าหรือ upgrade อย่างไรจึงจะ
minimize ค่าของ $\frac{\text{Actual cost}}{\text{Optimal cost}}$ 
ให้เหลือน้อยที่สุด ไม่ว่าโปรแกรมของนายกสิกรจะใช้เวลารันกี่ชั่วโมงก็ตาม?
\begin{enumerate}[label={$\Circle$}]
    \item นายกสิกรควร upgrade เป็นการคิดค่าบริการแบบ Flat rate ทันทีโดยไม่เสียค่าเช่ารายชั่วโมง
        (แปลว่านายกสิกรจะเสียเงิน 10,000 บาทเสมอ ไม่ว่าโปรแกรมจะรันกี่ชั่วโมง)
    \item นายกสิกรควรตัดสินใจเสียค่าเช่ารายชั่วโมงตลอดไป ไม่ควร upgrade เลย
    \item[\llap{*}$\Circle$] นายกสิกรควรเสียค่าเช่ารายชั่วโมงเป็นเวลาไม่เกิน $h$ ชั่วโมง 
        แล้วจึง upgrade หากโปรแกรมยังรันไม่เสร็จเมื่อเวลาผ่านไป $h$ ชั่วโมงพอดี \\ 
        (\textbf{หมายเหตุ:} หากเลือกตัวเลือกนี้ กรุณาระบุจำนวนเต็ม $h$ ดังกล่าวด้วย)
\end{enumerate}
