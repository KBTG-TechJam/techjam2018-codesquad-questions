\question{}

ในปัจจุบัน เงินตราที่นิยมใช้กันแพร่หลายในประเทศไทยประกอบไปด้วยเหรียญกษาปณ์หรือธนบัตรชนิดราคา\; 
1 บาท, 2 บาท, 5 บาท, 10 บาท, 20 บาท, 50 บาท, 100 บาท, 500 บาท และ 1000 บาท ตามลำดับ

สมมติว่าเราต้องการชำระยอดหนี้ก้อนหนึ่งซึ่งมีมูลค่า $d$ บาท
โดยมี\emph{{\hrsp}เป้าหมาย{\hrsp}}ว่าจะต้องใช้เงินตราเป็นจำนวน\uline{น้อยที่สุด}เพื่อชำระหนี้ดังกล่าว\uline{ให้พอดี}\;
สังเกตว่าเราสามารถใช้ \textbf{Greedy algorithm} ดังต่อไปนี้ เพื่อบรรลุ{\hrsp}\emph{เป้าหมาย}{\hrsp}ดังกล่าวได้\hrsp%
\sidenote{%
    หมายความว่า \textbf{Greedy algorithm} ให้ผลลัพธ์เป็น \emph{optimal} สำหรับเงินตราที่ระบุไว้ข้างต้น
} 
ไม่ว่ายอดหนี้ $d$ จะมีมูลค่ากี่บาทก็ตาม

\begin{quote}
    \textbf{Greedy algorithm.} เราจะเลือกเหรียญกษาปณ์หรือธนบัตรที่มีมูลค่ามากที่สุดที่เป็นไปได้ที่ไม่เกินยอดหนี้ นำไปหักจากยอดหนี้ 
    ทำเช่นนี้ไปเรื่อย ๆ จนกว่ายอดหนี้จะลดลงเหลือ 0 บาท

    ยกตัวอย่างเช่น หากเราต้องการชำระหนี้มูลค่า $d = \mathrm{94}$ บาท เราสามารถจ่ายด้วยเหรียญกษาปณ์หรือธนบัตรที่มีมูลค่า 
    $\mathrm{50+20+20+2+2}$ ตามลำดับ ซึ่งหมายความว่าเราใช้จำนวนเงินตรา 5 อัน ซึ่งน้อยที่สุดที่เพียงพอจะชำระหนี้ดังกล่าวพอดี
\end{quote}

\uline{อย่างไรก็ดี} สมมติว่าวันหนึ่ง\,ประเทศไทยจะเพิ่มเหรียญกษาปณ์หรือธนบัตรชนิดราคาใหม่จำนวน
1 ชนิดราคาเข้ามาในระบบ สังเกตว่า

\begin{itemize}[before*=\small]

\item ถ้าสมมติว่าประเทศไทยตัดสินใจเพิ่มเหรียญกษาปณ์ชนิดราคา 4 บาทเข้ามาในระบบ แล้ว \textbf{Greedy algorithm} 
ข้างต้นจะไม่รับประกันว่าจะให้ผลลัพธ์ที่ optimal เสมอไป\; 
(เช่น หากต้องการชำระเงิน $d = \mathrm{8}$ บาท \textbf{Greedy algorithm} จะจ่ายด้วยเหรียญ $\mathrm{5+2+1}$ บาทตามลำดับ 
แทนการใช้วิธี $\mathrm{4+4}$ บาทที่ใช้จำนวนเงินตราน้อยกว่า)

\item แต่ถ้าเราเพิ่มเหรียญกษาปณ์ชนิดราคา 3 บาทแล้ว \textbf{Greedy algorithm} จะยังคงให้จำนวนเงินตราที่ optimal 
อยู่ไม่ว่ายอดหนี้ $d$ จะมีมูลค่าเท่าใดก็ตาม

\end{itemize}

\uline{จงหาเงินตราที่มีชนิดราคาสูงที่สุด 1 ชนิดราคา} ที่เมื่อเพิ่มเข้ามาในระบบแล้วจะทำให้ 
\textbf{Greedy algorithm} ให้ผลลัพธ์ไม่เป็น optimal สำหรับยอดหนี้บางจำนวน\sidenote[][-\baselineskip]{%
    พร้อมทั้งยกตัวอย่างค้านว่า ยอดหนี้ $d$ มูลค่าเท่าใดที่ทำให้  \textbf{Greedy algorithm} ให้ผลลัพธ์ที่ไม่เป็น optimal
}
