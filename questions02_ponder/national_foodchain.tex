\question{}

จงพิจารณาระบบการถ่ายเทพลังงานในระบบห่วงโซ่อาหารของสัตว์กลุ่มหนึ่ง 
โดยที่สัตว์ทุกตัวในกลุ่มนี้มีโอกาสเป็นทั้งผู้ล่าและเหยื่อกับสัตว์ตัวอื่น ๆ ได้ทุกตัว
\begin{itemize}
\item กำหนดให้สัตว์ทุกตัวมี\uline{พลังงานสะสมเริ่มต้น}ในร่างกาย 1,000,000 KCal
\item เมื่อเกิดการล่าเหยื่อขึ้น หากสัตว์ A จับสัตว์ B เป็นอาหารแล้วพบว่า B จะเสียชีวิตไป 
    และ A จะได้รับถ่ายทอดพลังงานสะสม\uline{ครึ่งหนึ่ง}จาก B\hrsp%
    \sidenote{%
        \textbf{หมายเหตุ}\; เราสมมติว่าสัตว์แต่ละตัวจะไม่เสียพลังงานใด ๆ กับกิจกรรมอื่น ๆ เลย
        พลังงานจะหายไปจากระบบจากการที่เหยื่อถูกบริโภคเท่านั้น
    }
\item สัตว์ทุกตัวในระบบจะจับเหยื่อตัวอื่น ๆ เป็นอาหารได้ไม่เกิน 3 ตัว
\end{itemize}

หากในตอนแรกมีสัตว์ในระบบนิเวศนี้ทั้งสิ้น 1,000 ตัว แล้วจึงเกิดการล่ากันเองขึ้นจนเหลือสัตว์ที่อยู่รอดตัวเดียว 
สัตว์ตัวดังกล่าวนี้จะมีพลังงานสะสมตอนท้าย\uline{อย่างน้อยที่สุด}และ
\uline{อย่างมากที่สุด}กี่ KCal? ({\hrsp}ให้ตอบเป็นจำนวนเต็มโดย\uline{ปัดเศษทิ้ง}{\hrsp})
