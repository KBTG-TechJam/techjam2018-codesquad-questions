\question{}

กำหนดให้มี input ทั้งสิ้น 1 จำนวน ได้แก่จำนวนจริง $x$

จาก input ข้างต้นนี้ เป้าหมายคือการคำนวณค่าของ $x^n$ โดยที่ $n$ เป็นจำนวนเต็มบวก  
โดยใช้ operation การคูณเป็น\uline{จำนวนครั้งน้อยที่สุด} ภายใต้เงื่อนไขดังต่อไปนี้
\begin{itemize}
\item อนุญาตให้ใช้เฉพาะ operation การคูณ
\item อนุญาตให้นำผลคูณที่เกิดขึ้นก่อนหน้านั้น\uline{ระหว่างการคำนวณ}
    มาใช้เป็นตัวตั้งหรือตัวคูณของการคูณครั้งถัดไปได้\hrsp%
    \sidenote[][-\baselineskip]{%
        \textbf{ยกตัวอย่าง}\; สมมติว่าเราต้องการคำนวณ $x^n$ ในกรณีที่ $n = \mathrm{6}$ 
        เราจะใช้การคูณ\uline{น้อยที่สุด}เพียง 3 ครั้งเท่านั้น เขียนเป็นขั้นตอนวิธีได้ดังนี้
        \begin{flushleft}
            \quad\lstinline{r_1 := x * x      # => x^2} \\
            \quad\lstinline{r_2 := r_1 * r_1  # => x^4} \\
            \quad\lstinline{r_3 := r_1 * r_2  # => x^6}
        \end{flushleft}

        สังเกตว่า จากข้อมูล $x$ ที่เราทราบ เราจะคำนวณหาค่าของ $x^\mathrm{2}$, $x^\mathrm{4}$ 
        และ $x^\mathrm{6}$ ตามลำดับ\;
        นอกเหนือจากวิธีนี้ ยังมีวิธีอื่นอีก เช่น การคำนวณหา $x^\mathrm{2}$, $x^\mathrm{3}$ 
        และ $x^\mathrm{6}$ ตามลำดับ
    }
    (หมายความว่า เรามีการ\uline{จดบันทึก}ผลการคูณที่เกิดขึ้นทั้งหมด 
    คล้ายกับ history tape ในเครื่องคิดเลขของนักบัญชี)
\end{itemize}

\noindent
\textbf{\uline{โจทย์}} การคำนวณหาค่าของ $x^n$ ในกรณีที่ $n=\mathrm{125}$ 
(นั่นคือให้คำนวณค่าของ $x^\mathrm{125}$) จะต้องใช้ operation การคูณเป็นจำนวน\uline{น้อยที่สุด}กี่ครั้ง? 
และการคูณในแต่ละขั้นนั้นจะคำนวณ $x^{???}$ อะไรบ้าง\uline{ตามลำดับ}?\; (ให้ตอบมา 1 วิธี)
