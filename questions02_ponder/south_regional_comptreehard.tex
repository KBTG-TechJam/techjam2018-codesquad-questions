\question{}

กำหนดให้มี input ทั้งสิ้น 3 จำนวน ได้แก่จำนวนจริง $x$, $y$ และ $z$

จาก input ข้างต้นนี้ เป้าหมายคือการคำนวณค่าของนิพจน์ $ax + by + cz$
โดยที่ $a$, $b$ และ $c$ เป็นค่าคงที่จำนวนเต็มที่ไม่ติดลบ 
โดยใช้ operation การบวกเป็น\uline{จำนวนครั้งน้อยที่สุด}\;
ภายใต้เงื่อนไขดังต่อไปนี้

\begin{itemize}
    \item อนุญาตให้ใช้เฉพาะ operation การบวก
    \item อนุญาตให้นำผลบวกที่เกิดขึ้นก่อนหน้านั้น\uline{ระหว่างการคำนวณ}
        มาใช้เป็นตัวตั้งหรือตัวบวกของการบวกครั้งถัดไปได้\hrsp%
        (หมายความว่า เรามีการ\uline{จดบันทึก}ผลการบวกที่เกิดขึ้นทั้งหมด 
        คล้ายกับ history tape ในเครื่องคิดเลขของนักบัญชี)
\end{itemize}

\noindent
\textbf{\uline{ตัวอย่าง}} สมมติว่าเราต้องการคำนวณ $x + \mathrm{2}y + \mathrm{3}z$ ในเคสทั่วไป
เราอาจคำนวณตามลำดับในสมการ $x + y + y + z + z + z$ ซึ่งใช้การบวกทั้งสิ้น 5 ครั้ง\; 
แต่เนื่องจากเงื่อนไขอนุญาตให้นำผลบวกก่อนหน้ามาใช้งานได้ เราสามารถใช้การบวกน้อยที่สุดเพียง 4 ครั้งเท่านั้น
ซึ่งเขียนเป็นขั้นตอนวิธีได้ดังนี้
\begin{lstlisting}
r_1 := y + z       # => y + z
r_2 := r_1 + r_1   # => 2y + 2z
r_3 := r_2 + z     # => 2y + 3z
r_4 := r_3 + x     # => x + 2y + 3z
\end{lstlisting}

\noindent
\textbf{\uline{โจทย์}} การคำนวณหาค่าของ $x + \mathrm{4}y + \mathrm{9}z$ จาก input $x$, $y$ และ $z$ 
จะต้องใช้ operation การบวกเป็นจำนวน\uline{น้อยที่สุด}กี่ครั้ง?
