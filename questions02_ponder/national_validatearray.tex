\question{}

กำหนดให้ \lstinline{validate_array(A)} คือฟังก์ชันที่รับ input argument
เป็น 0-indexed array \lstinline{A} ของจำนวนเต็ม\;
และให้ output result เป็นค่า boolean ที่เป็น \lstinline{true} หรือ \lstinline{false} เท่านั้น\;
ฟังก์ชันดังกล่าว สามารถเขียนเป็น pseudocode ได้ดังนี้
\begin{fullwidth}
\vspace*{-\baselineskip}    
\begin{lstlisting}
function validate_array(A[0...n-1]):
    return (0 ≤ A[i] ≤ n-1 and A[i] == A[A[i]] for each i := 0 to n-1)  <%\SuppressNumber%>
           and (A[i-1] ≤ A[i] for each i := 1 to n-1)  <%\ReactivateNumber%>
end
\end{lstlisting}
\end{fullwidth}

\smallskip\noindent
ตัวอย่างของการเรียกใช้ฟังก์ชันข้างต้น
\begin{itemize}[itemsep=0pt]
\item \lstinline|validate_array(A = [0, 1, 1, 3])  # => true|
\item \lstinline|validate_array(A = [2, 2, 2, 2])  # => true|
\item \lstinline|validate_array(A = [1, 2, 3, 3])  # => false|
\item \lstinline|validate_array(A = [3, 1, 1, 3])  # => false|
\end{itemize}

\textbf{\uline{โจทย์}}\; จงหา\uline{จำนวนรูปแบบ}ทั้งหมดของ input array \lstinline|A| 
ที่มีจำนวนสมาชิก 20 ตัวที่ทำให้ \lstinline|validate_array(A)| คืนค่าออกมาเป็น \lstinline|true|?
