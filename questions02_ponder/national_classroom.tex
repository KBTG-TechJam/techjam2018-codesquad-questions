\question{}

ในโรงเรียนประถมศึกษาแห่งหนึ่ง มีห้องเรียนชั้น ป.1 อยู่ 2 ห้อง ได้แก่ห้องทานตะวัน และห้องกุหลาบ

เด็กชาย K เรียนอยู่ในห้องทานตะวัน และมีเพื่อนสนิทอยู่ 8 คน ได้แก่ A, B, C, D, E, F, G และ H\;\;
บางคนเรียนอยู่ห้องทานตะวันเช่นเดียวกับเด็กชาย K \;
ส่วนบางคนเรียนอยู่ห้องกุหลาบ คนละห้องกับเด็กชาย K

ต่อไปนี้คำบอกกล่าวของครูใหญ่ทั้งสิ้น 14 ประโยค

\begin{multicols}{2}
    \begin{enumerate}
        \item A กับ B เรียนอยู่คนละห้องกัน
        \item B กับ C เรียนอยู่คนละห้องกัน
        \item C กับ D เรียนอยู่คนละห้องกัน
        \item D กับ E เรียนอยู่คนละห้องกัน
        \item E กับ A เรียนอยู่คนละห้องกัน
        \item A กับ F เรียนอยู่คนละห้องกัน
        \item F กับ G เรียนอยู่คนละห้องกัน
        \item G กับ H เรียนอยู่คนละห้องกัน
        \item H กับ D เรียนอยู่คนละห้องกัน
        \item D กับ B เรียนอยู่คนละห้องกัน
        \item B กับ E เรียนอยู่คนละห้องกัน
        \item E กับ C เรียนอยู่คนละห้องกัน
        \item C กับ K เรียนอยู่คนละห้องกัน
        \item K กับ G เรียนอยู่คนละห้องกัน
    \end{enumerate}
\end{multicols}

ในบรรดา 14 ประโยคข้างต้น มี \uline{2 ประโยคที่ไม่เป็นความจริง} 
จงพิจารณาข้อมูลข้างต้นแล้วตอบคำถามต่อไปนี้
\begin{itemize}
\item ประโยคใดเป็นเท็จบ้าง?
\item ใครเรียนอยู่ห้องทานตะวันเช่นเดียวกับเด็กชาย K บ้าง?
\end{itemize}
