\question{\label{q:quickfire_central_audition_sortbetweenintro}}

กำหนดให้มีฟังก์ชันหนึ่ง \lstinline{sort_between(L, p, q)}    ซึ่งมีสเปกดังนี้ 
\begin{itemize}
    \item ข้อมูล input \lstinline{L} เป็น 0-indexed array ของจำนวนชุดหนึ่ง
    \item ข้อมูล input \lstinline{p} และ \lstinline{q} เป็น index 
        ภายใน array \lstinline{L} โดยที่มีเงื่อนไขว่า \\
        \lstinline{0 <= p <= q < L.length()}
    \item ข้อมูล output ของ \lstinline{sort_between(L, p, q)} 
        จะเป็น array ใหม่ ซึ่งเกิดจากการจัดเรียงจำนวนบางจำนวนใน array \lstinline{L} เดิม 
        ตั้งแต่ตำแหน่ง \lstinline{p} ถึงตำแหน่ง \lstinline{q} จากน้อยไปมาก
        และจำนวนในตำแหน่งอื่น ๆ นอกเหนือจากนี้ยังคงเดิม
\end{itemize}

\noindent
\textbf{\uline{ยกตัวอย่าง}}\; ถ้ากำหนดให้ \lstinline{L_0 = [4, 2, 7, 3, 8, 1, 5]}
แล้วเมื่อเรียกฟังก์ชัน \lstinline{sort_between(L_0, 2, 4)}  
จะได้ output เป็น \lstinline{[4, 2, 1, 3, 7, 8, 5]}

\medskip\noindent
\textbf{\uline{โจทย์}}\; สมมติว่าเรามี array \lstinline{L_1} 
ซึ่งประกอบไปด้วยจำนวนทั้งสิ้น 250 จำนวน โปรแกรมในข้อใดต่อไปนี้\hrsp\uline{ไม่รับประกัน}\hrsp%
ว่าสามารถเรียงลำดับจำนวนทุกจำนวนภายใน array ได้ทั้งหมด?
\begin{enumerate}[label={$\Circle$}]
\item \lstinline{sort_between(L_1, 0, 249)}
\item
    \lstinline{sort_between(L_1, 0, 199)} \\
    \lstinline{sort_between(L_1, 150, 249)} \\
    \lstinline{sort_between(L_1, 0, 199)}
\item 
    \lstinline{sort_between(L_1, 0, 199)} \\
    \lstinline{sort_between(L_1, 50, 249)} \\
    \lstinline{sort_between(L_1, 0, 149)}
\item 
    \lstinline{sort_between(L_1, 100, 249)} \\
    \lstinline{sort_between(L_1, 0, 149)} \\
    \lstinline{sort_between(L_1, 50, 249)}
\item 
    \lstinline{sort_between(L_1, 150, 249)} \\
    \lstinline{sort_between(L_1, 0, 199)} \\
    \lstinline{sort_between(L_1, 150, 249)}
\end{enumerate}
