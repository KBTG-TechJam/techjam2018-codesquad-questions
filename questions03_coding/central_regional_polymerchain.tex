\question{\bfseries Polymer Chain}

\subsection*{\sectionfont\upshape Background}

ในห้องปฏิบัติการวิจัยเคมีแห่งหนึ่ง มีการทดลอง Simulation การเปลี่ยนแปลงโครงสร้างสาย polymer 
ชื่อ polymer $K$ ซึ่งประกอบไปด้วยอะตอมหลากหลายชนิดเรียงตัวเป็นเส้นตรง

ในคอมพิวเตอร์ เราจะแทนอะตอมแต่ละชนิดด้วยอักขระ \verb|"A"| ถึง \verb|"Z"| 
และเราจะแทน polymer \lstinline{K} ที่มีความยาว $N$ ด้วยสตริงที่ประกอบไปด้วยอักขระ 
\verb|"A"| ถึง \verb|"Z"| ความยาว $N$ ตัว

ในห้องปฏิบัติการดังกล่าว ยังมีเครื่องจัดเรียง polymer ที่มีชื่อว่า
\begin{center}
    \lstinline{manipulate_polymer_once(K[0...N-1], p)} 
\end{center} 
ซึ่งมี Specification ในการทำงานดังต่อไปนี้

\begin{fullwidth}
    \vspace{2\baselineskip}
    \begin{quote}
        หาก input string \lstinline{K} ของฟังก์ชัน \lstinline{manipulate_polymer_once}
        คือลำดับของอักขระ \verb|K[0]|, \verb|K[1]|, \ldots, \verb|K[N-1]| ตามลำดับ 
        และ \lstinline{p} คือ index ของสตริง \lstinline{K} โดยที่ \lstinline{0 <= p <= N-1} 
        แล้ว ฟังก์ชันนี้จะ return ค่าสตริงซึ่งประกอบด้วยอักขระต่อไปนี้ตามลำดับ
        \[
            \underbrace{\text{\ttfamily K[p-1], K[p-2], \ldots, K[1], K[0],}}_\text{exists if $p > 0$}
            \text{\ttfamily K[p],} 
            \underbrace{\text{\ttfamily K[N-1], K[N-2], \ldots, K[p+2], K[p+1]}}_\text{exists if $p < N-1$}
        \]    
        ยกตัวอย่างเช่น ถ้า input ของ \lstinline{manipulate_polymer_once} ได้แก่ 
        \lstinline{K = "ASDFGHJKL"} และ \lstinline{p = 3}\;
        จะได้ output string เป็น \lstinline{"DSAFLKJHG"}\;
        สังเกตว่า \verb|F| จะอยู่ในตำแหน่งเดิมไม่เปลี่ยนแปลง แต่สตริงย่อยที่อยู่ข้างหน้าและข้างหลังจะถูกเรียงกลับหลัง
    \end{quote}
    \vspace{2\baselineskip}
\end{fullwidth}

ในการทำการทดลอง Simulation จริง เราจะนำ polymer สายหนึ่งมาจัดเรียงใหม่ไปเรื่อย ๆ 
อย่างต่อเนื่องเป็นจำนวน $M$ ครั้ง โดยปรับเปลี่ยน parameter ไปเรื่อย ๆ เช่น ถ้า polymer ตั้งต้นคือ 
\lstinline{K = "ASDFGHJKL"} และค่า parameter ของการจัดเรียง $M = 3$ ครั้งอย่างต่อเนื่องคือ 
\lstinline{p_1 = 3}, \lstinline{p_2 = 6}, \lstinline{p_3 = 0} เราจะได้ Polymer ผลลัพธ์เป็น
\begin{center}
    \texttt{ASDFGHJKL → DSAFLKJHG → KLFASDJGH → KHGJDSAFL}
\end{center}

\subsection*{\sectionfont\upshape Problem Statement}

จงเขียนโปรแกรมเพื่อทำ Simulation ของการจัดเรียง polymer สายหนึ่งด้วยลำดับของพารามิเตอร์ที่กำหนดให้ 
แล้วหาว่า polymer ผลลัพธ์สุดท้ายมีหน้าตาเป็นอย่างไร

\newpage
\subsection*{\sectionfont\upshape Program Specification}

โปรแกรมที่คุณเขียนจะต้องอ่านข้อมูลจาก stardard input 
และเขียนคำตอบลง standard output โดยข้อมูลจะมีฟอร์แมตดังต่อไปนี้

\bigskip\noindent
{\sectionfont\bfseries Input Format}
\begin{itemize}
\item บรรทัดที่ 1: มีจำนวนเต็มสองจำนวน $N$ และ $M$ คั่นด้วยช่องว่าง
\item บรรทัดที่ 2: มีสตริงความยาว $N$ ซึ่งระบุข้อมูลสาย Polymer เริ่มต้นก่อนการทดลอง
\item อีก $M$ บรรทัดถัดมา บรรทัดที่ $i+2$ จะมีค่า $p_i$ ซึ่งเป็น Parameter ของคำสั่งการจัดเรียง Polymer คำสั่งที่ $i$ ที่ต้องกระทำดับ Polymer $K$ ตามลำดับ
\begin{lstlisting}
N M
K[0...N-1]
p_1
p_2 <%\SuppressNumber\AlternateNumber{...}%>
    <%\AlternateNumber{M+2}%>
p_M <%\ReactivateNumber%>
\end{lstlisting}
\end{itemize}

\medskip\noindent
{\sectionfont\bfseries Output Format}\begin{itemize}
\item คำตอบประกอบตัวสตริง 1 ตัว ซึ่งก็คือสาย Polymer สุดท้ายหลังจากจัดเรียง Polymer ตามคำสั่งทั้งหมด $M$ ตามที่กำหนดให้ใน input
\end{itemize}

\subsection*{\sectionfont\upshape Data Example}
\begin{tabular}{p{0.45\linewidth}p{0.45\linewidth}}
\toprule
Example Input & Example Output \\
\midrule
\ttfamily\setstretch{0.8}
9 3 \newline
ASDFGHJKL \newline
3 \newline
6 \newline
0 &
\ttfamily\setstretch{0.8} KHGJDSAFL \\
\bottomrule
\end{tabular}

\subsection*{\sectionfont\upshape Constraints}

โปรแกรมของคุณจะถูกทดสอบกับ test cases สองชุด (เรียกว่าชุดเล็ก และชุดใหญ่)
\begin{itemize}
\item test cases ชุดเล็กจะมีเงื่อนไขว่า ความยาว polymer จะสอดคล้องกับเงื่อนไข \\
    $1 \leq N \leq 300$ 
    และจำนวนครั้งที่เรียกใช้งานเครื่องจัดเรียง polymer 
    สอดคล้องกับเงื่อนไข $0 \leq M \leq 30,\!000$
\item test cases ชุดใหญ่จะมีเงื่อนไขว่า ความยาว polymer จะสอดคล้องกับเงื่อนไข \\
    $1 \leq N \leq 300,\!000$ 
    และจำนวนครั้งที่เรียกใช้งานเครื่องจัดเรียง polymer 
    สอดคล้องกับเงื่อนไข $0 \leq M \leq 300,\!000$
\end{itemize}
