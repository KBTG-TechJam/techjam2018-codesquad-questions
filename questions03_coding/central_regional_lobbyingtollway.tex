\question{\bfseries Lobbying Tollway}

\subsection*{\sectionfont\upshape Background}

บริษัทขนส่งสินค้าแห่งหนึ่ง จำเป็นต้องวางแผนการลำเลียงส่งสินค้าระหว่างเมืองสองเมืองในดินแดนที่มีเมืองทั้งสิ้น $N$ เมือง 
และมีโครงข่ายของถนน $M$ สายที่เชื่อมเมืองเหล่านี้ให้เดินทางไปมาหาสู่กันได้ทั้งหมด 
เมืองแต่ละเมืองจะมีหมายเลข $1$ ถึง $N$ ส่วนถนนแต่ละสายจะมีหมายเลข $1$ ถึง $M$ ตามลำดับ

สำหรับแต่ละ $i=1,2,\ldots,M$ ถนนสายที่ $i$ จะเป็นถนนวิ่งทางเดียว (one-way road) 
ที่เชื่อมการเดินทางจากเมือง $u_i$ ไปยังเมือง $v_i$ เสมอ ($1 \leq u_i, v_i \leq N$) 
นอกจากนั้นอาจจะมีค่าผ่านทาง $p_i$ บาทที่คนใช้ถนนสายนี้ต้องจ่ายเพื่อใช้งาน ($p_i \geq 0$) 
นอกจากนั้น กำหนดว่าถ้าถนนสายไหนไม่มีค่าผ่านทาง นั่นแปลว่า $p_i = 0$

พึงทราบว่า อาจมีถนนวิ่งทางเดียวที่เชื่อมจากเมืองหนึ่งไปยังอีกเมืองหนึ่ง มากกว่า 1 สายก็ได้ 
นอกจากนั้นอาจมีถนนที่เชื่อมระหว่างเมืองสองเมือง ไป-กลับ โดยที่ถนนเหล่านี้เก็บค่าผ่านทางที่ไม่เท่ากันก็ได้

โดยปกตินั้น บริษัทนี้ได้สำรวจเส้นทางทั้งหมดที่เป็นไปได้ เพื่อใช้ลำเลียงสินค้าจากเมืองหมายเลข $1$ ไปยังเมืองหมายเลข $N$ 
โดยเส้นทางเหล่านี้ล้วนแต่เป็นเส้นทางที่เสียค่าผ่านทางรวมน้อยที่สุดทั้งสิ้น

ในเวลาต่อมา บริษัทนี้ต้องการเปิดเส้นทางการลำเลียงสินค้าเพิ่มขึ้นอย่างน้อย 1 เส้นทาง โดยมีเงื่อนไขต่อไปนี้
\begin{itemize}
\item บริษัทจะไปล็อบบี้กับผู้บริหารของเครือข่ายถนน เพื่อให้ลดค่าผ่านทางของถนนเพียง 1 สายเท่านั้น
\item ค่าผ่านทางใหม่นั้นจะติดลบไม่ได้
\item ค่าผ่านทางใหม่นั้นจะต้องลดลงจากค่าผ่านทางเดิม เป็นปริมาณเงินน้อยที่สุดเท่าที่เป็นไปได้
\item เส้นทางการลำเลียงสินค้าเดิมที่เคยสำรวจไว้จะต้องไม่กระทบ กล่าวคือเส้นทางเดิมแต่ละเส้นทางจะยังคงใช้งานได้เช่นเดิม และมีค่าผ่านทางรวมเท่าเดิม ไม่เพิ่มขึ้นหรือลดลง
\item จะต้องมีเส้นทางใหม่การลำเลียงสินค้าเกิดขึ้นอย่างน้อย 1 เส้นทาง และจะต้องไม่ซ้ำกันเส้นทางเดิมที่บริษัทเคยสำรวจไว้ และราคาค่าผ่านทางรวมของเส้นทางใหม่นี้จะต้องเท่ากับราคาค่าผ่านทางรวมของเส้นทางเดิมอื่น ๆ ของบริษัทด้วย
\end{itemize}

\subsection*{\sectionfont\upshape Problem Statement}

จงรับข้อมูลเครือข่ายถนนในดินแดนแห่งหนึ่ง รวมถึงค่าผ่านทางของถนนแต่ละสาย 
แล้วหาว่าบริษัทนี้จะต้องไปล็อบบี้เพื่อลดค่าผ่านทางของถนนสายใด 1 สาย 
และเป็นปริมาณเงินลดลงน้อยที่สุดเท่าใด จึงจะสามารถเปิดเส้นทางใหม่เพื่อใช้ลำเลียงสินค้าจากเมือง $1$ ไปเมือง $N$ ได้ 
โดยเส้นทางใหม่ที่เกิดขึ้นนี้จะมีค่าผ่านทางรวมถูกที่สุด และถูกเท่า~ๆ กับเส้นทางอื่น~ๆ ที่เคยมีการสำรวจมาก่อนหน้านี้แล้ว

หากมีถนนที่เป็นไปได้หลายสายที่สามารถล็อบบี้ให้ลดราคาลงเป็นปริมาณที่น้อยที่สุดได้ 
ให้ตอบหมายเลขของถนนทุกสายด้วย

\subsection*{\sectionfont\upshape Program Specification}

โปรแกรมที่คุณเขียนจะต้องอ่านข้อมูลจาก stardard input 
และเขียนคำตอบลง standard output โดยข้อมูลจะมีฟอร์แมตดังต่อไปนี้

\bigskip\noindent
{\sectionfont\bfseries Input Format}
\begin{itemize}
\item บรรทัดที่ 1: มีจำนวนเต็มสองจำนวน $N$ และ $M$ คั่นด้วยช่องว่าง
\item อีก $M$ บรรทัดถัดมา บรรทัดที่ $i+1$: จะมีจำนวนเต็มสามจำนวน $u_i, v_i, p_i$
    (คั่นด้วยช่องว่าง) ระบบข้อมูลของถนนหมายเลข $i$ ซึ่งเป็นถนนวิ่งทางเดียวจากเมืองหมายเลข $u_i$ 
    ไปยังเมืองหมายเลข $v_i$ และเก็บค่าผ่านทาง $p_i$ บาท
\begin{lstlisting}
N M
u_1 v_1 p_1
u_2 v_2 p_2 <%\SuppressNumber\AlternateNumber{...}%>
            <%\AlternateNumber{M+1}%>
u_M v_M p_M <%\ReactivateNumber%>
\end{lstlisting}
\textbf{หมายเหตุ:} ข้อมูล Input จะรับประกันว่า 
มีเส้นทางที่เชื่อมจากเมืองหมายเลข $1$ ไปเมืองหมายเลข $N$ เสมอ
\end{itemize}

\medskip\noindent
{\sectionfont\bfseries Output Format}
\begin{itemize}
\item บรรทัดที่ 1: จะต้องเขียนจำนวนเต็มสองจำนวน $D$ และ $K$ คั่นด้วยช่องว่างหนึ่งช่อง 
    โดยที่ $D$ จะระบุปริมาณค่าผ่านทางที่ลดลงน้อยที่สุดที่เป็นไปได้ และ $K$ คือจำนวนถนนทั้งหมดที่สามารถล็อบบี้ให้ลดค่าผ่านทางได้
\item อีก $K$ บรรทัดถัดมา แต่ละบรรทัดจะมีจำนวนเต็ม 1 จำนวน
    ซึ่งแต่ละจำนวนจะระบุหมายเลขถนนที่สามารถล็อบบี้ได้ นอกจากนั้น หมายเลขถนนทั้งหมดจะต้องเรียงจากน้อยไปมาก

    \textbf{หมายเหตุ:} ในกรณีที่บริษัทไม่สามารถใช้วิธีล็อบบี้ใด ๆ เพื่อเปิดเส้นทางใหม่ได้เลย 
    ให้ตอบว่า $D=0$ และ $K=0$ เป็นกรณีพิเศษ
\end{itemize}

\subsection*{\sectionfont\upshape First Data Example}
\begin{tabular}{p{0.45\linewidth}p{0.45\linewidth}}
\toprule
Example Input & Example Output \\
\midrule
\ttfamily\setstretch{0.8}
7 10 \newline
1 2 8 \newline
1 3 6 \newline
1 4 6 \newline
1 5 3 \newline
1 6 12 \newline
2 7 8 \newline
3 7 5 \newline
4 7 7 \newline
5 7 8 \newline
6 7 1 &
\ttfamily\setstretch{0.8}
2 3 \newline
3 \newline
5 \newline
8 \\
\bottomrule
\end{tabular}

\newpage\noindent
\textbf{อธิบายตัวอย่างที่ 1:} 
\begin{itemize}
\item จากตัวอย่างข้อมูลนี้ พบว่าจะมีเส้นทางลำเลียงที่ใช้ค่าผ่านทางรวมน้อยที่สุด 11 บาท 
    ซึ่งมี 2 เส้นทาง ได้แก่ (1) เส้นทางที่ใช้ถนนหมายเลข $2 \,\&\, 7$ 
    และอีกเส้นทางที่ใช้ถนนหมายเลข $4 \,\&\, 9$
\item หากเราล็อบบี้ให้มีการลดค่าผ่านทาง 2 บาท ให้แก่ถนน 1 สายในบรรดาถนน 3 สาย 
    สายได้ก็ได้ (ซึ่งได้แก่ถนนหมายเลข $3$, $5$ และ $8$) 
    แล้วจะทำให้มีเส้นทางลำเลียงสินค้าเส้นทางใหม่ที่ใช้เงินรวม 11 บาทเช่นกัน
\end{itemize}

\subsection*{\sectionfont\upshape Second Data Example}
\begin{tabular}{p{0.45\linewidth}p{0.45\linewidth}}
\toprule
Example Input & Example Output \\    
\midrule
\ttfamily\setstretch{0.8}
4 5 \newline
1 2 2 \newline
1 3 3 \newline
2 3 1 \newline
2 4 3 \newline
3 4 2 &
\ttfamily\setstretch{0.8} 
0 0 \\
\bottomrule
\end{tabular}

\subsection*{\sectionfont\upshape Constraints}

โปรแกรมของคุณจะถูกทดสอบกับ test cases สองชุด (เรียกว่าชุดเล็ก และชุดใหญ่)
\begin{itemize}
\item test cases ชุดเล็กจะมีเงื่อนไขว่า จำนวนเมืองทั้งหมดจะสอดคล้องกับเงื่อนไข \\
    $3 \leq N \leq 50$ และจำนวนถนนทั้งหมดจะสอดคล้องกับเงื่อนไข $1 \leq M \leq 2,\!000$
\item test cases ชุดใหญ่จะมีเงื่อนไขว่า จำนวนเมืองทั้งหมดจะสอดคล้องกับเงื่อนไข \\
    $3 \leq N \leq 100,\!000$ และจำนวนถนนทั้งหมดจะสอดคล้องกับเงื่อนไข $1 \leq M \leq 200,\!000$
\item สำหรับทุก test cases จะมีเงื่อนไขว่า ค่าผ่านทางเริ่มต้นของถนนทุกสายจะสอดคล้องกับเงื่อนไข 
    $0 \leq p_i \leq 5,\!000$
\end{itemize}
