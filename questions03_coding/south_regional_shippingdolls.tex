\question{\bfseries Shipping Dools}

\subsection*{\sectionfont\upshape Problem Statement}

โรงงานแห่งหนึ่งรับจ้างผลิตตุ๊กตาแบบสั่งทำพิเศษ\;
อยู่มาวันหนึ่งมีลูกค้า A มาติดต่อจ้างให้ผลิตตุ๊กตาทั้งสิ้น $N$ ตัว\; 
ตุ๊กตาแต่ละตัวมีหมายเลขกำกับ $i = {1, 2}, \ldots, N$\;
นอกจากนั้น ตุ๊กตาตัวที่ $i$ จะมีน้ำหนัก $w_i$ กรัม ซึ่งอาจเท่ากันหรือต่างกันก็ได้

เมื่อโรงงานแห่งนี้ผลิตตุ๊กตาเสร็จเป็นที่เรียบร้อยแล้ว โรงงานจะต้องขนส่งตุ๊กตาทั้งหมดนี้ให้ลูกค้า A\;\;
โรงงานสามารถเลือกขนส่งตุ๊กตา\uline{แต่ละตัว}ได้ 2 วิธี คือ 
(1) บรรจุตุ๊กตาลงในกล่องพัสดุที่จำกัดน้ำหนัก หรือ 
(2) บรรจุตุ๊กตาลงถุงกระสอบที่จำกัดจำนวนตุ๊กตา โดยมีเงื่อนไขว่า
\begin{itemize}
\item การขนส่งอาจใช้กล่องหลายใบก็ได้ กล่องแต่ละใบจุของน้ำหนักรวมไม่เกิน $L$ กรัม
\item การขนส่งสามารถใช้ถุงกระสอบได้เพียงถุงเดียว และใส่ตุ๊กตาได้ไม่เกิน $M$ ตัว (ไม่จำกัดน้ำหนัก)
\item หากตุ๊กตาหมายเลขที่ $s$ และตุ๊กตาหมายเลขที่ $t$ จะถูกบรรจุลงในกล่องใบเดียวกันแล้ว 
    ตุ๊กตาหมายเลขที่ $i$ แต่ละตัวซึ่งมีหมายเลขอยู่ระหว่าง $s$ กับ $t$ 
    จะต้องถูกบรรจุ\uline{ในกล่องใบ} \uline{เดียวกันด้วย} หรือจะต้องถูกบรรจุ\uline{ในถุงกระสอบ}เท่านั้น
\end{itemize}

\subsection*{\sectionfont\upshape Main Goal}

โรงงานต้องการขนส่งตุ๊กตาทั้งหมดให้ลูกค้า $A$ โดยใช้จำนวนกล่องให้น้อยที่สุด จะต้องใช้กล่องทั้งหมดกี่ใบ?

\subsection*{\sectionfont\upshape Program Specification}

โปรแกรมที่คุณเขียนจะต้องอ่านข้อมูลจาก stardard input 
และเขียนคำตอบลง standard output โดยข้อมูลจะมีฟอร์แมตดังต่อไปนี้

\bigskip\noindent
{\sectionfont\bfseries Input Format}
\begin{itemize}
\item บรรทัดที่ 1: มีจำนวนเต็มสามตัว $N, L, M$ คั่นด้วยช่องว่าง
\item อีก $N$ บรรทัดถัดมา บรรทัดที่ $i+1$ จะมีจำนวนเต็ม $w_i$ ระบุน้ำหนักของตุ๊กตาตัวที่ $i$
\begin{lstlisting}
N L M
w_1
w_2 <%\SuppressNumber\AlternateNumber{...}%>
    <%\AlternateNumber{N+1}%>
w_N <%\ReactivateNumber%>
\end{lstlisting}
\end{itemize}

\medskip\noindent
{\sectionfont\bfseries Output Format}
\begin{itemize}
\item คำตอบประกอบด้วยจำนวนเต็มตัวเดียว ซึ่งระบุจำนวนกล่องที่น้อยที่สุด%
    ที่สามารถใช้ขนส่งตุ๊กตาทั้งหมดตามเงื่อนไขโจทย์ข้างต้น
\end{itemize}

\newpage
\subsection*{\sectionfont\upshape Data Example}
\begin{tabular}{p{0.45\linewidth}p{0.45\linewidth}}
\toprule
Example Input & Example Output \\
\midrule
\ttfamily\setstretch{0.8}
6 5 1 \newline
1 \newline
2 \newline
3 \newline
2 \newline
1 \newline
4 &
\ttfamily\setstretch{0.8} 2 \\
\bottomrule
\end{tabular}

\medskip\noindent
\textbf{อธิบายตัวอย่าง:} หยิบตุ๊กตาตัวที่ $i={3}$ ซึ่ง $w_i = {3}$ ใส่ถุง 
จากนั้นหยิบตุ๊กตาตัวที่ ${1, 2, 4}$ ใส่กล่องใบแรก และตัวที่ ${5, 6}$ ใส่กล่องใบที่สอง

\subsection*{\sectionfont\upshape Constraints}

โปรแกรมของคุณจะถูกทดสอบกับ test cases สองชุด (เรียกว่าชุดเล็ก และชุดใหญ่)
\begin{itemize}
\item test cases ชุดเล็กจะมีเงื่อนไขว่า ${1} \leq N \leq {250}$
\item test cases ชุดใหญ่จะมีเงื่อนไขว่า ${1} \leq N \leq {2500}$
\item นอกจากนั้นกำหนดให้ ${1} \leq L \leq {10^8};\; {0} \leq M \leq N$ 
    และตุ๊กตาแต่ละตัวมีนำหนัก ${1} \leq w_i \leq L$
\end{itemize}
