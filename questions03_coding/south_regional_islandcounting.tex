\question{\bfseries Island Counting}

\subsection*{\sectionfont\upshape Background}

ในเกมย้อนยุคเกมหนึ่ง มีการวาดแผนที่ด้วย ASCII Art โดยประกอบไปด้วยอักขระ ASCII 
ที่เรียงตัวกันเป็นสี่เหลี่ยมผืนผ้าจำนวน $R$ แถว แถวละ $C$ ตัว

อักขระแต่ละตัวจะแทนช่อง 1 ช่องและมีค่าได้ 2 แบบ ได้แก่
\begin{itemize}
    \item ผืนดิน ซึ่งเราจะขอแทนด้วย `\verb|#|'
    \item ผืนน้ำ ซึ่งเราจะขอแทนด้วย `\verb|.|'
\end{itemize}

นอกจากนี้เราจะสมมติว่า รอบนอกสี่เหลี่ยมผืนผ้านี้จะ\uline{มีแต่น้ำทะเล}อันกว้างใหญ่ไพศาลและ\uline{ไม่มีผืนดิน}อยู่เลย

\bigskip\noindent
\textbf{\uline{ตัวอย่าง}}* นี่คือแผนที่ตัวอย่าง ซึ่งมีขนาด $R = {6}$ และ $C = {18}$
\begin{center}
    \vspace{0.25\baselineskip}
    \begin{centervrb}{18}
\begin{plainvrb}
................##
..#####...##......
.##...##..##...###
.#..#..#.......#..
.#....##..#.#..#.#
.######....#...#.#
\end{plainvrb}
    \end{centervrb}
    \vspace{0.25\baselineskip}
\end{center}

\subsection*{\sectionfont\upshape What is an island?}
จากแผนที่ในลักษณะข้างต้น เรานิยามภูมิประเทศทีของเกาะดังนี้%

\begin{itemize}
\item หากเราเริ่มต้นจากตำแหน่งผืนดินใด ๆ ในแผนที่ แล้วสามารถเดินทางไปยังผืนดินอื่น~ๆ 
    ข้างเคียงได้ด้วยการเดิน ขึ้น--ลง--ซ้าย--ขวา ไปเรื่อย~ๆ ได้ 
    ให้ถือว่าผืนดินที่เดินไปถึงทั้งหมดเหล่านั้นเป็นส่วนหนึ่งของเกาะเดียวกัน
\item มากไปกว่านั้น จากการเดินข้างต้น หากเราเริ่มเดินจนวกกลับมาที่ผืนดินเริ่มต้น 
    ให้ถือว่าบริเวณที่ถูกรายล้อมด้วยการเดินข้างต้นเป็นส่วนหนึ่งของเกาะเดียวกันเช่นกัน
    \begin{itemize}[before*=\small]
    \item สังเกตว่าอาจมีทะเลสาบที่ถูกรายล้อมด้วยผืนดินของเกาะเกาะหนึ่ง 
        ซึ่งผืนน้ำดังกล่าวจะถูกนับไปส่วนหนึ่งของเกาะนั้นด้วย
    \item ไม่เพียงแค่นั้น ผืนดินที่ซ้อนอยู่ภายในทะเลสาบดังกล่าว 
        ก็ยังถือว่าเป็นส่วนของเกาะภายนอกด้วย ไม่นับเป็นเกาะแยกต่างหาก
    \end{itemize}
\end{itemize}

\bigskip\noindent
\textbf{\uline{ตัวอย่าง}}* จากแผนที่ตัวอย่างเดิมข้างต้น 
เราสามารถเขียนใหม่โดยส่วนของเกาะเดียวกันเขียนด้วยตัวอักษรเดียวกันได้ดังนี้
(สังเกตได้ว่าแผนที่นี้จะมีเกาะทั้งสิ้น 8 เกาะ)
\begin{center}
    \vspace{0.25\baselineskip}
    \begin{centervrb}{18}
\begin{plainvrb}
................AA
..BBBBB...CC......
.BBBBBBB..CC...DDD
.BBBBBBB.......D..
.BBBBBBB..E.F..D.G
.BBBBBB....H...D.G
\end{plainvrb}
    \end{centervrb}
    \vspace{0.25\baselineskip}
\end{center}    

\subsection*{\sectionfont\upshape Problem Statement}

จากแผนที่ภายในเกมที่กำหนดให้ มีเกาะทั้งสิ้นกี่เกาะ?

\subsection*{\sectionfont\upshape Program Specification}

โปรแกรมที่คุณเขียนจะต้องอ่านข้อมูลจาก stardard input 
และเขียนคำตอบลง standard output โดยข้อมูลจะมีฟอร์แมตดังต่อไปนี้

\bigskip\noindent
{\sectionfont\bfseries Input Format}
\begin{itemize}[itemsep=0pt]
\item บรรทัดที่ 1: มีจำนวนเต็มสองตัว $R, C$ คั่นด้วยช่องว่าง
\item อีก $R$ บรรทัดถัดมา บรรทัดที่ $i+1$ จะมีสตริงความยาว $C$ 
    ที่ประกอบไปด้วย `\verb|.|' หรือ `\verb|#|' (ซึ่งบอกข้อมูลแถวนั้น ๆ ของแผนที่)
\begin{lstlisting}
R C
M[1,1...C]
M[2,1...C] <%\SuppressNumber\AlternateNumber{...}%>
           <%\AlternateNumber{R+1}%>
M[R,1...C] <%\ReactivateNumber%>
\end{lstlisting}
\textbf{หมายเหตุ:} ตัวแปร \verb|M| ข้างต้น คือแผนที่ซึ่งเขียนในรูปของ 1-indexed array สองมิติ
\end{itemize}

\medskip\noindent
{\sectionfont\bfseries Output Format}
\begin{itemize}
\item คำตอบประกอบด้วยจำนวนเต็มตัวเดียว ซึ่งระบุจำนวนเกาะในแผนที่ที่กำหนดให้
\end{itemize}

\newpage
\subsection*{\sectionfont\upshape Data Examples}
\begin{tabular}{p{0.45\linewidth}p{0.45\linewidth}}
\toprule
Example Input & Example Output \\
\midrule
\begin{centervrb}{18}
\begin{plainvrb}
6 18
................##
..#####...##......
.##...##..##...###
.#..#..#.......#..
.#....##..#.#..#.#
.######....#...#.#

\end{plainvrb} 
\end{centervrb} &
\verb|8| \\
\midrule
\begin{centervrb}{7}
\setstretch{0.8}
\begin{plainvrb}
7 7
.......
..####.
.#...#.
.#.#.#.
.#...#.
.#####.
.......

\end{plainvrb} 
\end{centervrb} &
\verb|2| \\
\bottomrule
\end{tabular}

\subsection*{\sectionfont\upshape Constraints}

โปรแกรมของคุณจะถูกทดสอบกับ test cases สองชุด (เรียกว่าชุดเล็ก และชุดใหญ่)
\begin{itemize}
\item test cases ชุดเล็กจะมีเงื่อนไขว่า ${1} \leq R, C \leq {200}$
\item test cases ชุดใหญ่จะมีเงื่อนไขว่า ${1} \leq R, C \leq {3000}$
\end{itemize}
