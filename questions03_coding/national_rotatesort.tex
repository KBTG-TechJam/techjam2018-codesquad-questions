\question{\bfseries Rotate Sort (Optimization Problem)}

\begin{quote}
    \em
    \vphantom{~}\llap{\adfdownleafleft\;}โจทย์ Coding
    ข้อนี้ ปรากฏในการแข่งขัน \techjam\ รอบชิงชนะเลิศระดับประเทศ
    โดยเป็นโจทย์ที่ผู้เข้าแข่งขันจะต้อง optimize เพื่อหาคำตอบที่ดีที่สุดในบรรดาผู้เข้าแข่งขันทั้งหมด
    หากโปรแกรมของผู้เข้าแข่งขันให้ผลลัพธ์ที่ดีมากเท่าใด ก็จะยิ่งมีโอกาสได้คะแนนสูงมากขึ้นเท่านั้น

    กติการที่ปรากฏในส่วนของ Submission and Scoring นั้นมีนัยยะสำคัญเฉพาะภายในการแข่งขันดังกล่าวเท่านั้น
    โปรดใช้เนื้อหาดังกล่าวเพื่อการอ้างอิงเท่านั้น
\end{quote}

\subsection*{\sectionfont\upshape Background}

กำหนดให้ operation ชื่อว่า \lstinline|rotate_3way| รับ input argument อยู่ 3 อย่าง ได้แก่
\begin{itemize}
\item array $A$ ของจำนวนเต็ม
\item ดัชนี $p$ และ $q$ ภายใน array $A$ ซึ่ง $1 \leq p \leq q \leq |A|$ \\
    (เมื่อ $|A|$ คือความยาวของ array $A$)
\end{itemize}

เมื่อเรียกใช้งาน \lstinline|rotate_3way(A, p, q)| จะได้ output value เป็น array ที่มีลักษณะเป็นดังนี้
\begin{fullwidth}
    \vspace*{-\baselineskip}
    \[
        \text{\ttfamily rotate\_3way(A, p, q)} =
        \left[
        \underbrace{\text{\ttfamily A[q+1], A[q+2], \ldots, A[N],}}_\text{can be empty}
        \text{\ttfamily A[p], A[p+1],\ldots,A[q],} 
        \underbrace{\text{\ttfamily A[1], A[2], \ldots, A[p-1]}}_\text{can be empty}
        \right]    
    \]    
\end{fullwidth}

\noindent
ยกตัวอย่างเช่น
\begin{itemize}[itemsep=0pt]
\item \lstinline{rotate_3way([5, 2, 1, 0], 2, 3) = [0, 2, 1, 5]}
\item \lstinline{rotate_3way([5, 2, 1, 0], 2, 2) = [1, 0, 2, 5]}
\item \lstinline{rotate_3way([5, 2, 1, 0], 1, 2) = [1, 0, 5, 2]}
\item \lstinline{rotate_3way([5, 2, 1, 0], 1, 4) = [5, 2, 1, 0]}
\end{itemize}

\subsection*{\sectionfont\upshape Problem Statement}

โจทย์ข้อนี้ โปรแกรมของผู้เข้าแข่งขันจะได้รับข้อมูลนำเข้าเป็น array $X$ ของจำนวนเต็มที่ไม่เรียงลำดับ 
กำหนดให้ $N = |X|$ คือความยาวของ array $X$

เป้าหมายของโปรแกรมของผู้เข้าแข่งขันคือ 
จะต้องเรียงลำดับจำนวนใน array $X$ โดยเรียกใช้งาน operation \lstinline|rotate_3way| 
เป็นจำนวนครั้งให้ได้น้อยที่สุดเท่าที่ผู้เข้าแข่งขันสามารถทำได้

กล่าวคือ เมื่อโปรแกรมของผู้เข้าแข่งขันได้รับข้อมูล array $X$ แล้วจะต้องระบุว่าจะเรียกใช้งาน 
\lstinline|rotate_3way| ทั้งสิ้นกี่ครั้ง และในแต่ละครั้ง จะกำหนดค่าดัชนี $p$ และ $q$ เท่าใด
ตามลำดับ

สำหรับเกณฑ์การให้คะแนนในข้อนี้ โปรดดูหัวข้อ Submission and Scoring

\subsection*{\sectionfont\upshape Program Specification}

โปรแกรมที่คุณเขียนจะต้องอ่านข้อมูลจาก stardard input 
และเขียนคำตอบลง standard output โดยข้อมูลจะมีฟอร์แมตดังต่อไปนี้

\bigskip\noindent
{\sectionfont\bfseries Input Format}
\begin{itemize}
\item บรรทัดที่ 1: จะมีจำนวนเต็มหนึ่งจำนวน ระบุ $N$ ซึ่งเป็นความยาวของ array $X$
\item อีก $N$ บรรทัดถัดมา บรรทัดที่ $i+1$ จะระบุจำนวน $X[i]$ ของ array $X$
\begin{lstlisting}
N
X[1]
X[2] <%\SuppressNumber\AlternateNumber{...}%>
     <%\AlternateNumber{N+1}%>
X[N] <%\ReactivateNumber%>
\end{lstlisting}
\end{itemize}

\medskip\noindent
{\sectionfont\bfseries Output Format}
\begin{itemize}
\item บรรทัดที่ 1: จะมีจำนวนเต็ม $K$ หนึ่งจำนวน ระบุจำนวนครั้งที่ operation \lstinline|rotate_3way| 
    จะถูกเรียกใช้งาน
\item อีก $K$ บรรทัดถัดมา บรรทัดที่ $j+1$ จะระบุจำนวนเต็มสองจำนวน $p_j$ และ $q_j$ 
    คั่นด้วยช่องว่ง ซึ่งเป็นดัชนี $p$ และ $q$ ของการเรียกใช้งาน operation \lstinline|rotate_3way| 
    ครั้งที่ $j$
\end{itemize}

\subsection*{\sectionfont\upshape Data Example}
\begin{tabular}{p{0.3\linewidth}p{0.3\linewidth}p{0.3\linewidth}}
\toprule
Example Input & Example Output A & Example Output B \\
\midrule
\ttfamily\setstretch{0.8}
4 \newline
5 \newline
2 \newline
1 \newline
0 &
\ttfamily\setstretch{0.8}
3 \newline
2 2 \newline
2 2 \newline
3 4 &
\ttfamily\setstretch{0.8}
4 \newline
2 4 \newline
2 4 \newline
3 3 \newline
3 3 \\
\bottomrule
\end{tabular}

\medskip\noindent
\textbf{อธิบายตัวอย่าง:} สังเกตว่าลำดับของ operation ที่แสดงในทั้งสองคำตอบข้างต้น 
สามารถทำให้ array  $X$ เรียงลำดับได้ถูกต้อง แต่คำตอบ A ใช้จำนวน operation น้อยกว่า B

\subsection*{\sectionfont\upshape Submission and Scoring}

สำหรับโจทย์ข้อนี้ พึงทราบว่า test case แต่ละอันจะมีคะแนนเต็ม 18 คะแนน นอกจากนั้น 
คะแนนรวมตอนท้ายสำหรับโจทย์ข้อนี้จะคิดจากค่าเฉลี่ยของคะแนนจากแต่ละ test case
จึงทำให้คะแนนรวมสูงสุดที่เป็นไปได้สำหรับข้อนี้คือ 18 คะแนน เช่นกัน 
(เศษทศนิยมของคะแนนจะถูกปัดทิ้งให้เหลือทศนิยม 6 ตำแหน่ง)

\bigskip\noindent
{\sectionfont\bfseries Minimum requirement} \\
สำหรับ test case แต่ละอัน โปรแกรมของผู้เข้าแข่งขันจะถือว่าให้ผลลัพธ์ที่\uline{ถูกต้อง} ก็ต่อเมื่อ
\begin{enumerate}
\item โปรแกรมให้ผลลัพธ์ตรงกับ Program Specification ที่โจทย์กำหนดให้
\item ผลลัพธ์ของโปรแกรมเรียกใช้งาน operation \lstinline|rotate_3way| 
    ที่ทำให้ array $X$ เรียงลำดับจากน้อยไปมากได้ถูกต้อง
\item ผลลัพธ์ของโปรแกรมเรียกใช้งาน operation \lstinline|rotate_3way| ไม่เกิน $100 N$ ครั้ง
\end{enumerate}

\noindent
\textbf{\uline{หมายเหตุ:}} หากโปรแกรมของคุณให้ผลลัพธ์ไม่ถูกต้อง จะได้ 0 คะแนนทันที\newline

\smallskip\noindent
{\sectionfont\bfseries Scoring scheme for correct answers} \\
เมื่อโปรแกรมของคุณให้ผลลัพธ์ที่ถูกต้อง กำหนดให้

\begin{itemize}
\item โปรแกรมของคุณเรียกใช้งาน operation \lstinline|rotate_3way| 
    เป็นจำนวน $K$ ครั้งสำหรับ test case นี้
\item ในบรรดาผู้เข้าแข่งขันทั้งหมด จำนวน operation ที่น้อยที่สุดที่มีผู้เข้าแข่งขันทำได้คือ $K^*$ ครั้ง 
    สำหรับ test case เดียวกัน
\end{itemize}
แล้วคะแนนของคุณสำหรับ test case อันนี้จะคำนวณจากสูตรดังต่อไปนี้
\[
    \begin{cases}
    18 &\quad\text{if $K = K^*$} \\
    6 + 9 \cdot\dfrac{K^*}{K} &\quad\text{if $K > K^*$}
    \end{cases}
\]

สังเกตว่าเมื่อโปรแกรมให้ผลลัพธ์ที่ถูกต้อง รับประกันว่าจะได้คะแนนอย่างน้อย 6 คะแนนสำหรับ test case นั้น ๆ\;
และกำหนดให้เศษที่เกิดขึ้นจากการหาร จะถูกปัดทิ้งให้เหลือทศนิยม 6 ตำแหน่ง

\subsection*{\sectionfont\upshape Constraints}

แต่ละ test case จะมีเงื่อนไขดังต่อไปนี้
\begin{itemize}
\item ความยาวของ array $X$ จะสอดคล้องกับเงื่อนไข $2 \leq N \leq 100$
\item จำนวนที่พบใน array $X$ จะอยู่ในช่วง $0$ ถึง $1000$ ซึ่งอาจมีบางจำนวนซ้ำกันได้
\end{itemize}
